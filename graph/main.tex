%%
%% This is file `sample-authordraft.tex',
%% generated with the docstrip utility.
%%
%% The original source files were:
%%
%% samples.dtx  (with options: `authordraft')
%% 
%% IMPORTANT NOTICE:
%% 
%% For the copyright see the source file.
%% 
%% Any modified versions of this file must be renamed
%% with new filenames distinct from sample-authordraft.tex.
%% 
%% For distribution of the original source see the terms
%% for copying and modification in the file samples.dtx.
%% 
%% This generated file may be distributed as long as the
%% original source files, as listed above, are part of the
%% same distribution. (The sources need not necessarily be
%% in the same archive or directory.)
%%
%% The first command in your LaTeX source must be the \documentclass command.
% \documentclass[sigconf,authordraft]{acmart}


%%%% As of March 2017, [siggraph] is no longer used. Please use sigconf (above) for SIGGRAPH conferences.

%%%% As of May 2020, [sigchi] and [sigchi-a] are no longer used. Please use sigconf (above) for SIGCHI conferences.

%%%% Proceedings format for SIGPLAN conferences 
% \documentclass[sigplan, anonymous, authordraft]{acmart}

%%%% Proceedings format for conferences using one-column small layout
\documentclass[sigconf]{acmart}

% NOTE that a single column version is required for submission and peer review. This can be done by changing the \doucmentclass[...]{acmart} in this template to 
% \documentclass[manuscript,screen]{acmart}

%%
%% \BibTeX command to typeset BibTeX logo in the docs

\setcopyright{acmcopyright}
\acmYear{2021}
\copyrightyear{2021}
\acmConference{ICPE '21}{April 19-23, 2021}{Rennes, France}
\acmPrice{TBA}
\acmDOI{TBA}
\acmISBN{TBA}

\newtheorem{theorem}{Theorem}

% to be able to draw some self-contained figs
% \usepackage{tikz}

% inlined bib file
\usepackage{filecontents}

\usepackage{graphicx}
\graphicspath{ {images/} }
\usepackage{tikz}
\usetikzlibrary{shapes.geometric}
\usetikzlibrary{arrows.meta,arrows}
\usepackage{subfigure}
% \usepackage{subcaption}

\usepackage{algorithm}
\usepackage[noend]{algpseudocode}
\usepackage{mdwmath}
\usepackage{mdwtab}
\usepackage[utf8]{inputenc}
\usepackage[english]{babel}
\usepackage{breqn}
\usepackage{pgfplots}
\usepackage{textcomp}


\usepackage{amsmath,amssymb}
\usepackage{bm}
\usepackage{mathtools}
\usepackage{amsthm}
\usepackage{xspace}
\usepackage{xcolor}
\usepackage[normalem]{ulem}
\usepackage{paralist}

\interdisplaylinepenalty=2500

\usepackage{dsfont}
\usetikzlibrary{patterns}
\usepackage{color,soul}


\newcommand{\sd}[1]{\textcolor{teal}{{\bf Sourav:} #1}}
\newcommand{\vinay}[1]{\textcolor{blue}{{\bf Vinay:} #1}}
\newcommand{\nitin}[1]{\textcolor{cyan}{{\bf Nitin:} #1}}
\newcommand{\ub}[1]{\textcolor{red}{{\bf UB:} #1}}
\newcommand{\todo}[1]{\textcolor{black}{\hl{{\bf Todo:} #1}}}

\newcommand{\vname}{{\sc CCR}}
\newcommand{\prot}{{\sc Renoir}}
\newcommand{\upd}{{\sc UpdateTables}}
\newcommand{\csk}{{\sc CanSkip}}
\newcommand{\etr}{Ethereum}
\newcommand{\geth}{{\sf go-ethereum}}
\newcommand{\parity}{{\sf parity-ethereum}}


\newcommand{\nodes}{{\bf N}}
\newcommand{\cset}{{\bf C}}
\newcommand{\bcn}{{\bf B}}
\newcommand{\bt}{{\bf T}}
\newcommand{\bu}{{\bf U}}
\newcommand{\bp}{{\bf P}}
\newcommand{\bs}{{\bf S}}
\newcommand{\txp}{{\mathbb T}}
\newcommand{\adv}{${\mathcal{A}}$\xspace}


\newcommand{\transValue}{{\sc TransValue}\xspace}
\newcommand{\versionTable}{{\sc LisValue}\xspace}
\newcommand{\transValueBold}{{\bf TransValue}\xspace}
\newcommand{\versionTableBold}{{\bf LisValue}\xspace}

\newcommand{\rs}{{\tt Rd}}
\newcommand{\ws}{{\tt Wr}}
\newcommand{\intv}{{\mathbb I}}
\newcommand{\tx}{{\tt tx}}
\newcommand{\key}{k}
\newcommand{\lat}{{\sf latency}}
\newcommand{\hv}{{\sf headValidTime}}
\newcommand{\bv}{{\sf blkValidTime}}
\newcommand{\bc}{{\sf blkCreateTime}}
\newcommand{\tv}{{\sf txnValidTime}}

\newcommand{\Tr}{{\sf True}}
\newcommand{\Fl}{{\sf False}}


\newcommand{\kt}{{\sf KeyTable}}
\newcommand{\txt}{{\sf TxTable}}
\newcommand{\keyList}{{\sf $L_{lis}$}}
\newcommand{\verList}{{\sf $ver_{lis}$}}
\newcommand{\var}{{\sf $A$}}
\newcommand{\keyTrans}{{\sf $L_{trans}$}}
\newcommand{\verTrans}{{\sf $ver_{trans}$}}
\newcommand{\hash}{{\sf Hash}}
\newcommand{\ver}{{\sf ver}}
\newcommand{\iwt}{{\sf isWrite}}
\newcommand{\la}{\langle}
\newcommand{\ra}{\rangle}
\newcommand{\idx}{{\sf index}}
\newcommand{\isv}{{\sf isValid}}
\newcommand{\ste}{{\sf state}}
\newcommand{\sz}{{\sf sz}}
\newcommand{\wsq}{{\sf WSQ}}

\newcommand{\tol}{TOL}
\newcommand{\tcg}{TCG}
\newcommand{\utx}{unaltered}
\newcommand{\siml}{similarity}
\newcommand{\Siml}{Similarity}

\theoremstyle{definition}
\newtheorem{definition}{Definition}[section]
\newtheorem{lemma}{Lemma}[section]


\AtBeginDocument{%
  \providecommand\BibTeX{{%
    \normalfont B\kern-0.5em{\scshape i\kern-0.25em b}\kern-0.8em\TeX}}}

%% Rights management information.  This information is sent to you
%% when you complete the rights form.  These commands have SAMPLE
%% values in them; it is your responsibility as an author to replace
%% the commands and values with those provided to you when you
%% complete the rights form.
% \setcopyright{acmcopyright}
% \copyrightyear{2018}
% \acmYear{2018}
% \acmDOI{10.1145/1122445.1122456}

%% These commands are for a PROCEEDINGS abstract or paper.
% \acmConference[Woodstock '18]{Woodstock '18: ACM Symposium on Neural
%   Gaze Detection}{June 03--05, 2018}{Woodstock, NY}
% \acmBooktitle{Woodstock '18: ACM Symposium on Neural Gaze Detection,
%   June 03--05, 2018, Woodstock, NY}
% \acmPrice{15.00}
% \acmISBN{978-1-4503-XXXX-X/18/06}


%%
%% Submission ID.
%% Use this when submitting an article to a sponsored event. You'll
%% receive a unique submission ID from the organizers
%% of the event, and this ID should be used as the parameter to this command.
%%\acmSubmissionID{123-A56-BU3}

%%
%% The majority of ACM publications use numbered citations and
%% references.  The command \citestyle{authoryear} switches to the
%% "author year" style.
%%
%% If you are preparing content for an event
%% sponsored by ACM SIGGRAPH, you must use the "author year" style of
%% citations and references.
%% Uncommenting
%% the next command will enable that style.
%%\citestyle{acmauthoryear}

% \pagestyle{headings}
% \pagenumbering{arabic}
% \setcounter{page}{1}

%%
%% end of the preamble, start of the body of the document source.
\begin{document}

%%
%% The "title" command has an optional parameter,
%% allowing the author to define a "short title" to be used in page headers.
\title{\prot: Accelerating Blockchain Validation  using State Caching}

%%
%% The "author" command and its associated commands are used to define
%% the authors and their affiliations.
%% Of note is the shared affiliation of the first two authors, and the
%% "authornote" and "authornotemark" commands
%% used to denote shared contribution to the research.
% \author{Ben Trovato}
% \authornote{Both authors contributed equally to this research.}
% \email{trovato@corporation.com}
% \orcid{1234-5678-9012}
% \author{G.K.M. Tobin}
% \authornotemark[1]
% \email{webmaster@marysville-ohio.com}
% \affiliation{%
%   \institution{Institute for Clarity in Documentation}
%   \streetaddress{P.O. Box 1212}
%   \city{Dublin}
%   \state{Ohio}
%   \postcode{43017-6221}
% }

% \author{Lars Th{\o}rv{\"a}ld}
% \affiliation{%
%   \institution{The Th{\o}rv{\"a}ld Group}
%   \streetaddress{1 Th{\o}rv{\"a}ld Circle}
%   \city{Hekla}
%   \country{Iceland}}
% \email{larst@affiliation.org}

% \author{Lars Th{\o}rv{\"a}ld}
% \affiliation{%
%   \institution{The Th{\o}rv{\"a}ld Group}
%   \streetaddress{1 Th{\o}rv{\"a}ld Circle}
%   \city{Hekla}
%   \country{Iceland}}
% \email{larst@affiliation.org}

% \author{Valerie B\'eranger}
% \affiliation{%
%   \institution{Inria Paris-Rocquencourt}
%   \city{Rocquencourt}
%   \country{France}
% }

% \author{Aparna Patel}
% \affiliation{%
%  \institution{Rajiv Gandhi University}
%  \streetaddress{Rono-Hills}
%  \city{Doimukh}
%  \state{Arunachal Pradesh}
%  \country{India}}

\author{Nitin Awathare}
\affiliation{%
  \institution{IIT Bombay}}
\email{nitina@iitb.ac.in}
 
\author{Sourav Das}
\affiliation{%
  \institution{UIUC}}
\email{souravd2@illinois.edu}

\author{Vinay Joseph Ribeiro}
\affiliation{%
  \institution{IIT Bombay}}
\email{vinayr@iitb.ac.in}

\author{Umesh Bellur}
\affiliation{%
  \institution{IIT Bombay}}
\email{ubellur@iitb.ac.in}

% \author{John Smith}
% \affiliation{\institution{The Th{\o}rv{\"a}ld Group}}
% \email{jsmith@affiliation.org}

% \author{Julius P. Kumquat}
% \affiliation{\institution{The Kumquat Consortium}}
% \email{jpkumquat@consortium.net}

%%
%% By default, the full list of authors will be used in the page
%% headers. Often, this list is too long, and will overlap
%% other information printed in the page headers. This command allows
%% the author to define a more concise list
%% of authors' names for this purpose.
\renewcommand{\shortauthors}{Anon.}

%%
%% The abstract is a short summary of the work to be presented in the
%% article.
\begin{abstract}
% \vinay{I would structure the abstract as follows: (1) What do current PoW chains do: local block creation using cached transactions (creation phase), PoW (mining phase), validation of received block (validation phase). (2) Problems with long validation phase. (3) What do state of the art solutions do and their shortcomings. (4) We do an extensive measure which shows overlap between transactions in creation and validation. (5) Renoir idea (6) How much Renoir reduces validation time in Ethereum. (7) ?? Benefits of using Renoir in production blockchains -- this is not clear to me. We seem to be avoiding this issue, but the reviewer is likely to wonder if by using Renoir we solve all the problems mentioned in point (2).}



A Blockchain system such as \etr\ is a peer to peer network where each node works in three phases: {\em creation, mining,} and {\em validation} phases. In the creation phase, it executes a subset of  locally cached transactions to form a new block. In the mining phase, the node solves a cryptographic puzzle (Proof of Work - PoW) on the block it formed. % If it successfully solves the puzzle, it has "mined" a block which it broadcasts and then restarts the creation phase. However, if while solving the PoW puzzle it receives a 
%a block from another peer, it terminates the mining phase. 
On receiving a block from another peer, it starts the validation phase, where it executes the transactions in the received block in order to validate it. Since transactions depend on the state that previously executed transactions have created, a node must validate each newly arrived block before creating a new block on top of it.
%, and then starts the creation phase.
%process and restart it from the beginning after validating the newly arrived block. Lastly, the node solves a cryptographic puzzle (Proof of Work - PoW) on the block it created - a process termed {\em mining}.  
%validates a block arriving from another peer by executing the block's transactions and updating the state of smart contracts that these transactions operate on - a process termed {\em block validation}. 
%A Blockchain system such as \etr\ is a peer to peer network where each node works in three phases: First, it will select and execute a subset of transactions initiated by users to form a new block - a process termed {\em block creation}. Second, the node validates a block arriving from another peer by executing the block's transactions and updating the state of smart contracts that these transactions operate on - a process termed {\em block validation}. 
% This is a computationally intensive process that slows down the creation of new blocks thereby impacting the throughput of a blockchain network.\nitin{can we remove this sentence?}  
%The arrival of a block from another peer during block creation causes a node to abandon it's block creation process and restart it from the beginning after validating the newly arrived block. Lastly, the node solves a cryptographic puzzle (Proof of Work - PoW) on the block it created - a process termed {\em mining}.  
%Since transactions depend on the state that previously executed transactions have created, a node must validate each newly arrived block.
% (validation builds new state that is needed for executing transactions to be included in the next block). 
A long block validation time lowers the system's overall throughput and brings the well known Verifier's dilemma 
% (whether to skip validating the newly arrived block at the risk of accepting invalid block) 
into play. Additionally, this leads to wasted mining power utilization (MPU).

% \vinay{Earlier, we mentioned other approaches (off-chain, parallel execution) and said ours is a new one. So people will expect us to compare our work with these. I have commented out the sentence about other works. We have to make sure we justify why we do not compare to earlier work somewhere in the paper.}
%Existing approaches to reducing the validation time either use optimistic parallel execution of transactions or off-chain methods that employ untrusted computational nodes. However, the fact that 
%A key finding of our paper, 
%is that nodes re-execute many transactions that have previously been run during block creation has not been exploited. 


In this work, we present Renoir a novel mechanism that 
caches state from transaction execution during the block creation 
phase and reuses it to enable nodes to skip (re)executing these transactions during block validation. Renoir artifact consists of two parts: First, the extensive measurement from the production Ethereum network to check the extent of redundancy in the transaction execution during block creation and validation phase. Second, Evaluation of Renoir using different metrics(Throughput, Mining Power Utilization and Validation time) on a 50 node testbed mimicking the top 50 Ethereum miners.

% **********************************************************************************
% Through extensive measurement of 2000 nodes from the production \etr\ network
%  we find  that during block validation,  nodes {\em redundantly}
%  execute more than 80\% of the transactions in greater than 
%  75\% of the blocks they receive - this points to significant potential to save time and computation during block validation. 
 
% Motivated by this, we present \prot, a novel mechanism that caches state from transaction execution during the block creation phase and reuse it to enable nodes to skip (re)executing these transactions during block validation. Nodes on the \etr\ network can incorporate the \prot\ mechanism to speed up validation.
% % \vinay{We need to rewrite the following. RENOIR is something that can be run on Ethereum. It is not an alternative to Ethereum} 
% % \ub{Nodes on the \etr\ network can incorporate the \prot\ mechanism to speed up validation.}
% Our detailed evaluation of \prot\ on a 50 node testbed mimicking the top 50 \etr\ miners illustrates that the reduction in block validation time due to \prot\ is greater than  90\% for blocks with computationally intensive transactions. Such blocks reduce the throughput and MPU of \etr\ but, barely has any effect in case of \prot. Specifically, the inclusion of computationally intensive transactions, reduces the throughput of \etr\ from 35326 tx/hour to 24716 tx/hour and MPU from 96\% to 67\%. Furthermore, to measure the gains with our system on the production
% \etr\ network, we deploy a node running \prot\ on the production
% \etr\ network. Our measurement illustrates that \prot\ reduces
% the block validation time by as much as 50\%. 
% ************************************************************************************
% \vinay{is it true for ALL blockchains? Maybe we need to limit the scope to "PoW blockchains such as Bitcoin and Ethereum".} 

% blockchain network executes the transactions of each received block to ensure that they are valid and to update the state of smart contracts impacted by the transactions - a computationally intensive process \vinay{why is it intensive? Current Bitcoin and Ethereum have very short validation times. So how is it intensive?} termed Block Validation.  High block validation time lowers the overall throughput of the system and brings the Verifier's dilemma into play.  And further, as we prove, this leads to wasted mining power utilization.
    
% Existing approaches to reducing the validation time either use optimistic parallel execution of transactions or off-chain methods that employ untrusted computational nodes. \vinay{It is not clear that block creation happens before block validation.} However, the fact that a node executes many of the same transactions as a part of the block creation and block validation process has not been exploited. 
%     %  \vinay{Does not \prot\ also need some additional resources and is more likely to fail due to dependencies across transactions? If so, we need a better way to contrast our method to the earlier work} However these techniques have limited scalability and 
%      % require extra-trust assumptions between nodes.
%     %  are likely to fail in the presence of dependency across 
%     %  transactions in a block.
%      Our  monitoring of approximately 2000 nodes from the production \etr\ network
%      illustrates that during block validation, nodes redundantly
%      execute more than 80\% of the transactions in more than 
%      75\% of the  blocks - this points to significant potential in savings the block validation time. 
%     %  \etr\ nodes re-execute more than 75\% of the total transactions
%     %  from the block creation period. 
%      Motivated by this, we present
%      \prot, a novel mechanism that caches state variables and the accessed version of it from the block creation phase to enable nodes
%      to skip (re)executing these transactions during block validation. 
%      %We corroborate the design of \prot\ with a formal theoretical proof of its correctness. 
%     %  \ub{What is the subtelity of the \prot mechanism? What's clever about this?}
%      In order to measure the
%      gains with our system, we deploy a node running \prot\ on the production
%      \etr\ network. Our measurement illustrates that \prot\ reduces
%      the block validation time by 50\%. Our detailed evaluation of
%      \prot\ atop a 50 node test bed mimicking the top 50 \etr\ miners, \vinay{why is top 50 miners good? what percentage of hashing power do they have?}
%     %  \vinay{is the mining pool term correct? If not, remove it}
%      illustrates that 
%     %  with increasing block validation, 
%      the reduction
%      in block validation time due to \prot\ is greater than
%     %  \vinay{Is it always 90 percent? If not, under what conditions is it so high?} 
%      90\% for 
%      blocks with computationally intensive transactions - blocks with high validation time.\vinay{It is not clear what the actual benefits are if a) only a single miner uses Renoir (either in current Renoir or in the CIT sceanario) -- will that miner create more fraction of blocks compared to if it did not use Renoir, (b) if ALL miners use Renoir -- will the transaction throughput or gas limit go up?}
\end{abstract}


% \maketitle
% \input{sections/artifact}
% \input{sections/introduction}
% \input{sections/background}
% \input{sections/casestudy}
% % \input{sections/overview}
% \input{sections/design}
% \input{sections/analysis}
% !TEX root = ../main.tex

\pgfplotsset{small,label style={font=\fontsize{8}{9}\selectfont},legend style={font=\fontsize{7}{8}\selectfont},height=4.2cm,width=1.3\textwidth}
\begin{figure*}[t!]
    \begin{minipage}{0.295\textwidth}
    \subfigure[MPU]{
    \label{fig:mpu-system}
    \begin{tikzpicture}
    \begin{axis}[
        % ybar,
        enlargelimits=0.25,
        % bar width=9.9pt,
        % /pgfplots/ybar=0pt,
        legend columns=4,
        legend pos=north east,
        xlabel={$\tau/\intv$},
        ylabel= {Mining Power util.},
        symbolic x coords={0.011, 0.122, 0.139, 0.205, 0.253},
        grid=minor,
        xtick={0.011, 0.122, 0.139, 0.205, 0.253},
        ymin=55,
        ymax=135,
        ]
        % \addplot [fill=blue!30!white] [error bars/.cd, y explicit,y dir=both,] table [x=gasUsage, y=forkRate, y error=cfiFork,  col sep=comma] {data/eth-minefrac-renoir.csv}; 
        % \addplot [fill=red!30!white] [error bars/.cd, y explicit,y dir=both,] table [x=gasUsage, y=forkRate, y error=cfiFork, col sep=comma] {data/renoir-minefrac-renoir.csv};

        \addplot [blue] [error bars/.cd, y explicit,y dir=both,] table [x=gasUsage, y=forkRate, y error=cfiFork,  col sep=comma] {data/eth-minefrac-renoir.csv}; 
        \addplot [red] [error bars/.cd, y explicit,y dir=both,] table [x=gasUsage, y=forkRate, y error=cfiFork, col sep=comma] {data/renoir-minefrac-renoir.csv};
        
        \addlegendentry{$Ethereum$}
        \addlegendentry{$Renoir$}
        
    \end{axis}
    \end{tikzpicture}
    }
    \end{minipage}\hfill \hfill \hfill \hfill \hfill \hfill
    \begin{minipage}{0.295\textwidth}
    \subfigure[Throughput]{
    \label{fig:throughput}
    \begin{tikzpicture}
    \begin{axis}[
        % ybar,
        enlargelimits=0.25,
        % bar width=9.9pt,
        % /pgfplots/ybar=0pt,
        legend columns=4,
        legend pos=north east,
        xlabel={$\tau/\intv$},
        ylabel= {Throughput(tx/hour)},
        symbolic x coords={0.011, 0.122, 0.139, 0.205, 0.253},
        grid=minor,
        xtick={0.011, 0.122, 0.139, 0.205, 0.253},
        ymin=25000,
        ymax=45000,
        ]
        \addplot [blue] [error bars/.cd, y explicit,y dir=both, error bar style={color=black},] table [x=ratio, y=throughput, y error=cfi, col sep=comma] {data/eth-throughput.csv}; 
        \addplot [red] [error bars/.cd, y explicit,y dir=both, error bar style={color=black},] table [x=ratio, y=throughput, y error=cfi, col sep=comma] {data/renoir-throughput.csv};

        \addlegendentry{$Ethereum$}
        \addlegendentry{$Renoir$}
        
    \end{axis}
    \end{tikzpicture}
    }
    \end{minipage}\hfill \hfill \hfill \hfill \hfill \hfill

    
    \setlength{\abovecaptionskip}{6pt}
    \setlength{\belowcaptionskip}{-1pt}
    \caption{Mining power utilization and throughput of \etr\ and \prot\ with a confidence interval of 95\%
    measured in our private network with varying block creation-arrival 
    ratio}
    \label{fig:ethRenoirThroughput}
\end{figure*}    


\pgfplotsset{small,label style={font=\fontsize{8}{9}\selectfont},legend style={font=\fontsize{7}{8}\selectfont},height=4.2cm,width=1.3\textwidth}
\begin{figure*}[t!]
    \begin{minipage}{0.295\textwidth}
    \subfigure[\etr]{
    \label{fig:mpu-eth}
    \begin{tikzpicture}
    \begin{axis}[
        ybar,
        enlargelimits=0.25,
        bar width=5.9pt,
        /pgfplots/ybar=0pt,
        legend columns=4,
        legend pos=north east,
        xlabel={Block creation-arrival ratio},
        ylabel= {Mining Power util.},
        symbolic x coords={0.011, 0.122, 0.205},
        grid=minor,
        xtick={0.011, 0.122, 0.205},
        ymin=55,
        ymax=135,
        ]
        \addplot [pattern = north east lines] [error bars/.cd, y explicit,y dir=both,] table [x=gasUsage, y=frac1, y error=cfi1,  col sep=comma] {data/eth-minefrac-renoir.csv}; 
        \addplot [fill=red!30!white] [error bars/.cd, y explicit,y dir=both,] table [x=gasUsage, y=frac2, y error=cfi2, col sep=comma] {data/eth-minefrac-renoir.csv};
        \addplot [pattern = north west lines] [error bars/.cd, y explicit,y dir=both,] table [x=gasUsage, y=frac3, y error=cfi3,  col sep=comma] {data/eth-minefrac-renoir.csv}; 
        \addplot [fill=blue!30!white] [error bars/.cd, y explicit,y dir=both,] table [x=gasUsage, y=frac4, y error=cfi2, col sep=comma] {data/eth-minefrac-renoir.csv};
        \addplot [fill=blue!70!white] [error bars/.cd, y explicit,y dir=both,] table [x=gasUsage, y=frac5, y error=cfi3, col sep=comma] {data/eth-minefrac-renoir.csv};
        % \addplot [fill=red!60!white] [error bars/.cd, y explicit,y dir=both,] table [x=gasUsage, y=frac6, y error=cfi1, col sep=comma] {data/eth-minefrac-renoir.csv};
         \addplot [fill=green!30!white] [error bars/.cd, y explicit,y dir=both,] table [x=gasUsage, y=frac7, y error=cfi1, col sep=comma] {data/eth-minefrac-renoir.csv};
         \addplot [fill=green!60!white] [error bars/.cd, y explicit,y dir=both,] table [x=gasUsage, y=frac8, y error=cfi4, col sep=comma] {data/eth-minefrac-renoir.csv};
        \addlegendentry{$n_1$}
        \addlegendentry{$n_2$}
        \addlegendentry{$n_3$}
        \addlegendentry{$n_4$}
        \addlegendentry{$n_5$}
        % \addlegendentry{$n_6$}
        \addlegendentry{$n_6$}
        \addlegendentry{$n_7$}
    \end{axis}
    \end{tikzpicture}
    }
    \end{minipage}\hfill \hfill \hfill \hfill \hfill \hfill
    \begin{minipage}{0.295\textwidth}
    \subfigure[\prot]{
    \label{fig:mpu-prot}
        \begin{tikzpicture}
    \begin{axis}[
        ybar,
        enlargelimits=0.25,
        bar width=5.9pt,
        /pgfplots/ybar=0pt,
        legend columns=4,
        legend pos=north east,
        xlabel={Block creation-arrival ratio},
        % ylabel= {Mining Power util.},
        symbolic x coords={0.011, 0.139, 0.253},
        grid=minor,
        xtick={0.011, 0.139, 0.253},
        ymin=55,
        ymax=135,
        ymajorticks=false,
        ]
        \addplot [pattern = north east lines] [error bars/.cd, y explicit,y dir=both,] table [x=gasUsage, y=frac1, y error=cfi1,  col sep=comma] {data/renoir-minefrac-renoir.csv}; 
        \addplot [fill=red!30!white] [error bars/.cd, y explicit,y dir=both,] table [x=gasUsage, y=frac2, y error=cfi2, col sep=comma] {data/renoir-minefrac-renoir.csv};
        \addplot [pattern = north west lines] [error bars/.cd, y explicit,y dir=both,] table [x=gasUsage, y=frac3, y error=cfi3,  col sep=comma] {data/renoir-minefrac-renoir.csv}; 
        \addplot [fill=blue!30!white] [error bars/.cd, y explicit,y dir=both,] table [x=gasUsage, y=frac4, y error=cfi4, col sep=comma] {data/renoir-minefrac-renoir.csv};
        \addplot [fill=blue!70!white] [error bars/.cd, y explicit,y dir=both,] table [x=gasUsage, y=frac5, y error=cfi5, col sep=comma] {data/renoir-minefrac-renoir.csv};
        % \addplot [fill=red!60!white] [error bars/.cd, y explicit,y dir=both,] table [x=gasUsage, y=frac6, y error=cfi1, col sep=comma] {data/eth-minefrac-renoir.csv};
         \addplot [fill=green!30!white] [error bars/.cd, y explicit,y dir=both,] table [x=gasUsage, y=frac7, y error=cfi7, col sep=comma] {data/renoir-minefrac-renoir.csv};
         \addplot [fill=green!60!white] [error bars/.cd, y explicit,y dir=both,] table [x=gasUsage, y=frac8, y error=cfi8, col sep=comma] {data/renoir-minefrac-renoir.csv};
        \addlegendentry{$n_1$}
        \addlegendentry{$n_2$}
        \addlegendentry{$n_3$}
        \addlegendentry{$n_4$}
        \addlegendentry{$n_5$}
        % \addlegendentry{$n_6$}
        \addlegendentry{$n_6$}
        \addlegendentry{$n_7$}
    \end{axis}
    \end{tikzpicture}
    }
    \end{minipage}\hfill
    \begin{minipage}{0.295\textwidth}
    \subfigure[\prot, Block creation-arrival ratio=0.253]{
        \label{fig:mpu-prot-high}
            \begin{tikzpicture}
    \begin{axis}[
        ybar,
        enlargelimits=0.25,
        bar width=5.9pt,
        /pgfplots/ybar=0pt,
        legend columns=4,
        legend pos=north east,
        xlabel={\Siml\ in \%},
        symbolic x coords={93, 75, 50},
        grid=minor,
        xtick={93, 75, 50},
        ymin=55,
        ymax=135,
        ymajorticks=false,
        ]
        \addplot [pattern = north east lines] [error bars/.cd, y explicit,y dir=both,] table [x=gasUsage, y=frac1, y error=cfi1,  col sep=comma] {data/similarity-minefrac-renoir.csv}; 
        \addplot [fill=red!30!white] [error bars/.cd, y explicit,y dir=both,] table [x=gasUsage, y=frac2, y error=cfi2, col sep=comma] {data/similarity-minefrac-renoir.csv};
        \addplot [pattern = north west lines] [error bars/.cd, y explicit,y dir=both,] table [x=gasUsage, y=frac3, y error=cfi3,  col sep=comma] {data/similarity-minefrac-renoir.csv}; 
        \addplot [fill=blue!30!white] [error bars/.cd, y explicit,y dir=both,] table [x=gasUsage, y=frac4, y error=cfi4, col sep=comma] {data/similarity-minefrac-renoir.csv};
        \addplot [fill=blue!70!white] [error bars/.cd, y explicit,y dir=both,] table [x=gasUsage, y=frac5, y error=cfi5, col sep=comma] {data/similarity-minefrac-renoir.csv};
        % \addplot [fill=red!60!white] [error bars/.cd, y explicit,y dir=both,] table [x=gasUsage, y=frac6, y error=cfi1, col sep=comma] {data/eth-minefrac-renoir.csv};
         \addplot [fill=green!30!white] [error bars/.cd, y explicit,y dir=both,] table [x=gasUsage, y=frac7, y error=cfi7, col sep=comma] {data/similarity-minefrac-renoir.csv};
         \addplot [fill=green!60!white] [error bars/.cd, y explicit,y dir=both,] table [x=gasUsage, y=frac8, y error=cfi8, col sep=comma] {data/similarity-minefrac-renoir.csv};
        \addlegendentry{$n_1$}
        \addlegendentry{$n_2$}
        \addlegendentry{$n_3$}
        \addlegendentry{$n_4$}
        \addlegendentry{$n_5$}
        % \addlegendentry{$n_6$}
        \addlegendentry{$n_6$}
        \addlegendentry{$n_7$}
    \end{axis}
    \end{tikzpicture}
    }
    \end{minipage}\hfill

    \setlength{\abovecaptionskip}{6pt}
    \setlength{\belowcaptionskip}{-1pt}
    \caption{Mining power utilization of the first 7 nodes of \etr\ and \prot\ with a confidence interval of 95\%  measured in our private network with varying block creation-arrival ratio and varying \siml}
    \label{fig:mpu}
\end{figure*}

% \section{\prot\ Evaluation}
% \label{sec:evaluations}
% We implemented \prot\ on the open-source Go-Ethereum client version ${\tt 1.9.3}$ and measured its performance under two different setups described below. 

% \subsection{Experimental Setup}
% \label{sub:expt-setup}
% %
% \emph{First}, we deploy a \prot\ equipped node to the \etr\ mainnet and
% investigate the extent of reduction in the block validation time due to 
% \prot. The node had one 2.19GHz dual-core CPU, 8 GB RAM, and 6.4TB NVMe SSD. 

% \emph{Second}, we create a private blockchain network on 50 Oracle Virtual
% Machines to observe the effect of varying \emph{block creation time} on both
% \etr\ and \prot. In this setup each VM is equipped with one 2.19GHz dual-core
% CPU, 8 GB RAM and 128GB HDD. All VMs were running ubuntu 16.04 with download
% and upload bandwidth of 1 GBps and  100 MBps, respectively. Each VM runs one
% blockchain network node. Throughout the experiment, we have controlled the
% block mining difficulty so as to take 15 seconds to solve the Proof-of-Work
% puzzle to mine the block. This implies that the average block inter-arrival 
% in our \etr\ experiment is 15 seconds + block propagation delay + block
% creation time + block validation 
% time and the last quantity is replaced with the \prot\ validation time for
% \prot. With this experimental setup, we compare \prot\ and \etr\ on metrics 
% we define below. 
% % Also, the result for each metrics is plotted with the confidence interval of 95\%.


% \vspace{1mm}
% \noindent
% {\bf Nodes, network delays and topology.}
% We assign mining power to each node in our 50 node setup in 
% accordance with the distribution of mining power of the top 50 
% miners of the \etr\ network~\cite{ethMining}. The top 50 miners 
% (by mining power) contribute to around 99.98\% of total mining 
% power of the real \etr\ network, with the most powerful miner 
% controlling ${\sim}33\%$ of the total mining power. 
% %
% Also, we use the geographical location of these top 50 \etr\ 
% miners from~\cite{ethMining} to mimic the location of our 50 
% nodes. We ensure that the inter-node latency (using Linux ${\tt tc}$ 
% command) between any pair of nodes is in line with the ping delay 
% corresponding to the geographic locations of the nodes~\cite{pingDelay}
% in effect mimicking the delays between real nodes of the \etr\ mainnet. 

% In line with the topology of the Bitcoin network where the degree 
% of a node follows the power-law distribution~\cite{bitcoinTopology} 
% we design the topology of our experimental setup as follows: Each 
% node connects to a random set of other nodes such that the degree 
% of the node follows the power-law distribution.

% \vspace{1mm}
% \noindent
% {\bf Applications tested.} 
% We evaluate both \prot\ and the \etr\ by deploying three types of 
% smart contracts, each implementing quicksort, 2D matrix multiplication,
% and loop iteration with basic arithmetic operations. 
% Throughout the experiment, we maintain an average of ${\sim}165$
% transactions per block which is the average number of transactions 
% in a \etr\ block. Thus, whenever required, we vary the block 
% creation time by varying the time it takes for a node to execute
% each of these transactions. 

% \vspace{1mm}
% \noindent
% {\bf Parameters and Metrics.}
% The \emph{block creation-arrival ratio} is the ratio of the block 
% creation time to the average block inter-arrival time and is an important parameter in the performance evaluation of a blockchain
% system.
% In particular, a high block-creation interval ratio indicates
% that the system has high throughput if it(high block-creation interval) is the result of including more number of transactions in the block.  
% Thus, we investigate the effect of increasing block 
% creation-arrival ratio on \emph{mining power utilization},  
% block validation time and throughput. Throughput is the number of main chain transactions processed per unit time.
% Mining power utilization of a node is the ratio of the 
% number of blocks mined by the node that eventually makes it to
% the main chain to that of the total number of blocks mined by 
% the node. This indicates the extent to which mining was 
% successful - the blocks that do not make it to the main chain
% represent wasted effort.  


% \subsection{Experiments and Results}
% \label{sub:results}
%
\begin{figure}[t!]
   \centering
    \pgfplotsset{footnotesize,height=4.5cm, width=0.55\linewidth}
    \begin{tikzpicture}
    \begin{axis}[
        legend pos=north east,
        legend columns=2,
        ylabel=Block validation time (ms),
        xlabel=Block height,
        % symbolic x coords={9951489, 9951989, 9952489},
        grid=minor,
        xtick={9951489, 9951989, 9952489},
        ymax=300,
        ]
        \addplot [line width=0.75, mark=none ,mark size=1.2pt, red] table [x=blockNum, y=exTime, col sep=comma] {data/avc.csv};
        \addplot [line width=0.75, mark=none ,mark size=1.2pt, blue] table [x=blockNum, y=exTime, col sep=comma] {data/ethereumData.csv}; 
        \addlegendentry{with \prot}
        \addlegendentry{without \prot}
    \end{axis}
    \end{tikzpicture}
    \caption{Validation time of 1000 real \etr\ blocks at two nodes, one of which is equipped with 
    \prot\ and the other is not.}
    \label{fig:tau-eth}
\end{figure}
%
\begin{figure}[t!]
    \centering
    \pgfplotsset{footnotesize,height=4.5cm, width=0.55\linewidth}
    \begin{tikzpicture}
    \begin{axis}[
        legend pos=north west,
        ylabel=Block validation time(ms),
        xlabel=Block creation-interval ratio,
        % symbolic x coords={0.011, 0.122, 0.205, 0.260},
        grid=minor,
        xtick={0.011, 0.122, 0.139, 0.205, 0.253},
        ]
        \addplot table [x=delay, y=avgHonestProcTime, col sep=comma] {data/ethNoSkip.csv};
        \addplot table [x=delay, y=avgHonestProcTime, col sep=comma] {data/RenoirNoSkip.csv}; 
 
        \addlegendentry{\etr}
        \addlegendentry{\prot}

    \end{axis}
    \end{tikzpicture}
    \caption{Time taken by the node to validate a received block with and without \prot\ for varying block creation-interval ratio.}
    \label{fig:high-tau-local}
\end{figure}

% {\bf Reduction in block validation time.}
% Figure~\ref{fig:tau-eth} illustrates the reduction in block
% validation time as a result of using \prot\ in \etr\ public
% network. Specifically, observe that without \prot\ a \etr\ 
% host node takes ${\sim}200$ milliseconds to validate 
% a received block, whereas a node equipped with \prot\ only 
% takes ${\sim}100$ milliseconds, hence, a 50\% reduction
% in block validation time. 
% The reduction in block validation time is lower than our 
% estimated \siml\ of more than 80\%~(\S\ref{sub:findings}), 
% because the node spends additional time to decide whether
% to skip a transaction or not. 
% Similarly, Figure~\ref{fig:high-tau-local} illustrates the 
% reduction in block validation time we observe on our second 
% setup (private 50 node network) with varying block
% creation-interval ratio with 
% ${\sim}$93\% \siml. In particular, we find that for high
% block creation-interval ratio, nodes equipped with \prot\ 
% only spend a tiny fraction of a second to validate received
% block. On the other hand, nodes equipped with \etr\ takes 
% over a few seconds to validate the block. 

% \vspace{1mm}
% \noindent
% {\bf Mining power utilization.} 
% A low mining power utilization indicates a high degree of wasted 
% computation. In Figure~\ref{fig:mpu-system}, we observe that the overall mining power 
% utilization significantly drops with increase in block creation-arrival ratio, i.e., 
% with high block creation/validation time. In Figure~\ref{fig:mpu-eth}, we
% observe that with the increase in block creation-arrival ratio, the mining power 
% utilization of the first seven nodes in \etr\ drops significantly.
% Furthermore, this decrease is more for nodes with lower 
% mining power. 
% The reason behind this decrease is that, with high block 
% creation and validation time, blocks take longer to 
% propagate as nodes only forward those blocks for which it has
% validated all its ancestor blocks. This results
% in a high fork rate 
% % \ub{the work fork is mentioned for the first time here. Needs a sentence in the background to explain what it is}
% in the network and hence low mining 
% power utilization. 

% In Figure~\ref{fig:mpu-system} we observe that mining power utilization of the system in \prot\ 
% remains unaffected even for high block creation-interval ratio of 0.253, unlike \etr.
% Figure~\ref{fig:mpu-prot} illustrates the mining power 
% utilization of first seven \prot\ nodes (ordered by mining power) 
% in our experiment with 93\% \siml. 
% Unlike \etr, in \prot\
% % , even for high block creation-interval 
% % ratio of 0.253, 
% the mining power utilization of nodes remains 
% unaffected. This is due the fact that despite high block creation 
% time, as illustrated in Figure~\ref{fig:high-tau-local} the 
% block validation time in \prot\ is very small. 
% %
% We also evaluate \prot\ by varying the \siml\ and measure its effect on 
% mining power utilization for block creation-interval ratio of 0.253. 
% Figure~\ref{fig:mpu-prot-high} illustrates our findings. Observe that even 
% with 50\% \siml, mining power utilization of nodes are better than
% mining power utilization of \etr\ at higher block creation-arrival 
% ratio of 0.205. This illustrate that \prot\ achieves better, mining 
% power utilization and is robust against variations in \siml. 

% \vspace{1mm}
% \noindent
% {\bf Throughput} 
% Figure~\ref{fig:throughput} illustrates that throughput of the system in \etr\ declines 
% with the increase in block creation-arrival ratio. On the other hand, we observe that 
% higher block creation-interval ratio barely affects the throughput of \prot. This is 
% because of the higher fork rate, as observed in figure~\ref{fig:mpu-system}, 
% which then delays the extension of the main chain. Note that the increase in block creation-interval ratio in our experiment is a result of the inclusion of the transactions that require more amount of computation, in the blocks and not the increase in the number of transactions. Later will give the scope to increase the throughput further but, at the cost of an increase in block size. 

% the result of including the transactions that require more computation.

% \begin{figure}[t!]
%     \centering
%     \pgfplotsset{footnotesize,height=4.5cm, width=0.85\linewidth}
%     \begin{tikzpicture}
%     \begin{axis}[
%         legend pos=north west,
%         % legend style={font=\fontsize{0.1}{0.2}},
%         % bar width=4.8pt,
%         % /pgfplots/ybar=0pt,
%         % ybar,
%         % ymin=0,
%         ymax=1.9,
%         % xmax=4,
%         xlabel={Average number of table lookup per transaction},
%         ylabel= {Number of blocks(normalized)},
%         % grid=minor,
%         ] 
%       \addplot [line width=0.75, mark=*, blue] table [x=tableLookup, y=cumulative, col sep=comma] {data/tableLookUp.csv};
%         \addplot [line width=0.75, mark=*, red] table [x=tableLookup, y=cumulative, col sep=comma] {data/tableLookUpOurExperiment.csv};

%         \addlegendentry{Table lookup in Ethereum}
%         \addlegendentry{Table lookup in Renoir experiment}
%     \end{axis}
%     \end{tikzpicture}
    
%     \caption{Variation of average number of table lookup per transaction per block}
%     \label{fig:tableLookUpCom}
% \end{figure}




% \begin{figure}[t!]
%     \centering
%     \pgfplotsset{footnotesize,height=4.5cm, width=0.85\linewidth}
%     \begin{tikzpicture}
%     \begin{axis}[
%         legend pos=north west,
%         % legend style={font=\fontsize{0.1}{0.2}},
%         % bar width=4.8pt,
%         % /pgfplots/ybar=0pt,
%         % ybar,
%         % ymin=0,
%         ymax=1.4,
%         % xmax=4,
%         xlabel={Dependency},
%         ylabel= {Number of blocks(normalized)},
%         % grid=minor,
%         ] 
%       \addplot [line width=0.75, mark=triangle ,mark size=1.2pt, blue] table [x=dependency, y=cumulative, col sep=comma] {data/dependencyData.csv};

%         \addlegendentry{Table lookup in Ethereum}
%         \addlegendentry{Table lookup in Renoir experiment}
%     \end{axis}
%     \end{tikzpicture}
    
%     \caption{Dependency of transactions inside the block defined as the ratio of number of edges in the conflict graph to the maximum possible edges in the conflict graph ($n(n-1)/2$).}
%     \label{fig:similarityGraph}
% \end{figure}


% \vspace{1mm}
% \noindent
% {\bf Mining power distribution.} 
% We assign mining power to each node in our 50 node setup in line with the
% distribution of mining power of the top 50 miners of the real \etr\
% network~\cite{ethMining}. The top 50 miners (by mining power) contribute to
% around 99.98\% of total mining power of the real \etr\ network, with the 
% most powerful miner controlling ${\sim}33\%$ of the total mining power. 

% Assuming block generation follows the Poisson process with rate $\lambda$, 
% we ensure that each node generates blocks at the rate $\lambda/h$ where 
% $h$ is mining power assigned to the node. 

% \begin{table}[b!]
%     \begin{center}
%         \begin{tabular}{c c c c c c c}
%         \hline
%         32.98 & 16.16 & 15.06 & 5.72 & 5.67 & 4.41 & 4.14 \\
%         3.53 & 2.61 & 1.84 & 1.34 & 1.32 & 1.25 & 1.05 \\
%         \hline
%         \end{tabular}
%     \end{center}
%     \label{fig:mining fraction}
%     \caption{Percentage of mining power controlled by top 14
%     miners in descending order. }
%     % For 
%     % e.g. first miner controls $32.98\%$ of the mining power 
%     % and $14^{\rm th}$ miner controls $1.05\%$ of the mining 
%     % power. \ub{this is a caption, not a paragraph remove all this - its visible from the data anyway.}}
%     \label{tab:mining}
% \end{table}

% \input{sections/related}
% \input{sections/discussion}

% \begin{acks}
% To Robert, for the bagels and explaining CMYK and color spaces.
% \end{acks}

%%
%% The next two lines define the bibliography style to be used, and
%% the bibliography file.
% \bibliographystyle{ACM-Reference-Format}
% \bibliography{sample-base}


\end{document}
\endinput
%%
%% End of file `sample-authordraft.tex'.

